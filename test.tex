\documentclass{article}
\usepackage[T1]{fontenc}
\usepackage[polish]{babel}
\usepackage[utf8]{inputenc}
\usepackage{graphicx}
\usepackage[labelformat=empty]{caption}
\graphicspath{ {images/} }
\usepackage{contour}

\usepackage[margin=2cm]{geometry}

\usepackage[square,numbers]{natbib}
\bibliographystyle{abbrvnat}

\addto\captionspolish{\renewcommand{\refname}{\normalsize{Bibliografia}}}
\renewcommand{\bibname}{\normalsize{Bibliografia}}

\title{Projekt w \LaTeX{}\\
\hfill\break
{\Huge{\contour{black}{\color{white}{POL}\color{red}{SKA}}}}}
\author{\large Aleksandra Panek}

\date{\selectlanguage{polish}\today}

\begin{document}

\maketitle
\hfill\break
\hfill\break
\hfill\break
\hfill\break
\large{\tableofcontents}
\large{\listoftables}

\pagebreak
\raggedright
\section{\textbf{\textit{Podstawowe informacje}}}

\subsection{\underline{Położenie}}
\hfill\break
\large{\textbf{Polska, Rzeczpospolita Polska}} – państwo unitarne w Europie Środkowej, położone między Morzem Bałtyckim na północy a Sudetami i Karpatami na południu, w przeważającej części w dorzeczu Wisły i Odry. Od północy Polska graniczy z Rosją (z jej obwodem kaliningradzkim) i Litwą, od wschodu z Białorusią i Ukrainą, od południa ze Słowacją i Czechami, od zachodu z Niemcami. Większość północnej granicy Polski wyznacza wybrzeże Morza Bałtyckiego.\cite{wiki}

\hfill\break
\subsection{\underline{Ludność}}
\hfill\break

\subsubsection{\textbf{Informacje}}
\hfill\break
Polska jest zamieszkana przez 37 672 367 ludzi (2020), zajmuje pod względem liczby ludności \textbf{38.} miejsce \textit{na świecie}, a \textbf{5.} w \textit{Unii Europejskiej.} \\
Polska podzielona jest na 16 województw. Jej największym miastem i jednocześnie stolicą jest Warszawa. Inne metropolie to Kraków, Łódź, Wrocław, Poznań, Gdańsk, Szczecin.\\ 
Polska jest krajem jednolitym etnicznie – 97\% ludności deklaruje narodowość polską.\\

\subsubsection{\textbf{Cytaty o Polsce i Polakach}}
\hfill\break
\textit{\textbf{''Jestem Polakiem – więc całą rozległą stroną swego ducha żyję życiem Polski, jej uczuciami i myślami, jej potrzebami, dążeniami i aspiracjami. Im więcej nim jestem, tym mniej z jej życia jest mi obce i tym silniej chcę, żeby to, co w mym przekonaniu uważam za najwyższy wyraz życia stało się własnością całego narodu. ,,\cite{roman}}}\\
\hfill\break
\hfill\break
\textit{\textbf{''Patriotyzm nie polega na przekrzykiwaniu się kto Polskę bardziej kocha. Rzecz w tym , aby po cichu, z zaciętymi zębami, nieco pochylonym karkiem, ale z podniesioną głową żyć w niej i nie uciekać. ,,\cite{kamien}}}\\
\hfill\break
\hfill\break
\textit{\textbf{''Polak, chociaż stąd między narodami słynny,\\
Że bardziej niźli życie kocha kraj rodzinny,\\
Gotów zawżdy rzucić go, puścić się w kraj świata,\\
W nędzy i poniewierce przeżyć długie lata,\\
Walcząc z ludźmi i z losem, póki mu śród burzy\\
Przyświeca ta nadzieja, że Ojczyźnie służy. ,,\cite{pan}}}
\pagebreak

\section{\textbf{\textit{Rząd}}}

\subsection{Partie polityczne}
\hfill\break
Partie polityczne w sejmie polskim od 2019:\\
\begin{itemize}
	\item Prawo i Sprawiedliwość (PiS)
	\item Koalicja Obywatelska (KO)
	\item Sojusz Lewicy Demokratycznej (SLD)
	\item Polskie Stronnictwo Ludowe (PSL)
	\item Konfederacja Wolność i Niepodległość (Konfederacja)
	\item Mniejszość Niemiecka
\end{itemize}
\hfill\break
\subsection{Prezydenci}

\begin{table}[h!]
\caption{\large{Prezydenci III RP (od roku 1989)}}
\centering
\Large
\begin{tabular}{ |l|l|l|  }
 \hline
 Lp. &Imię i nazwisko &Okres sprawowania\\
 \hline
 1 &Wojciech Jaruzelski &31.12.1989 - 22.12.1990\\
 \hline
 2 &Lech Wałęsa &22.12.1990 - 22.12.1995\\
 \hline
 3 &Aleksander Kwaśniewski &23.12.1995 - 23.12.2005\\
  \hline
 4 &Lech Kaczyński &23.12.2005 - 10.04.2010\\
  \hline
 5 &Bronisław Komorowski &06.08.2010 - 06.08.2015\\
  \hline
 6 &Andrzej Duda &06.08.2015 - obecnie\\
  \hline
\end{tabular}
\end{table}

\begin{figure}[h!]
\begin{center}
 \includegraphics[scale=0.7]{andrzej}
\end{center}
  \caption{Zdjęcie obecnego prezydenta Polski - Andrzeja Dudy, źródło: \cite{zrodlo}}
\end{figure}

{\footnotesize{\bibliography{Biblio}}}

\end{document}