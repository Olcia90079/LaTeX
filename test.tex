\documentclass{article}
\usepackage[T1]{fontenc}
\usepackage[polish]{babel}
\usepackage[utf8]{inputenc}
\usepackage{graphicx}
\usepackage[labelformat=empty]{caption}
\graphicspath{ {images/} }
\usepackage{contour}

\usepackage[square,numbers]{natbib}
\bibliographystyle{abbrvnat}

\title{Projekt w \LaTeX{}\\
{\Huge{\contour{black}{\color{white}{POL}\color{red}{SKA}}}}}
\author{\large Aleksandra Panek}

\date{\selectlanguage{polish}\today}

\begin{document}

\maketitle

\tableofcontents
\listoftables

\pagebreak

\section{Podstawowe informacje}

\subsection{Położenie}
\textbf{Polska, Rzeczpospolita Polska} – państwo unitarne w Europie Środkowej, położone między Morzem Bałtyckim na północy a Sudetami i Karpatami na południu, w przeważającej części w dorzeczu Wisły i Odry. Od północy Polska graniczy z Rosją (z jej obwodem kaliningradzkim) i Litwą, od wschodu z Białorusią i Ukrainą, od południa ze Słowacją i Czechami, od zachodu z Niemcami. Większość północnej granicy Polski wyznacza wybrzeże Morza Bałtyckiego.\cite{wiki}


\subsection{Ludność}
\subsubsection{Informacje}
\subsubsection{Cechy charakterystyczne}


\section{Rząd}

\subsection{Partie polityczne}



\subsection{Prezydenci}

\begin{figure}[h!]
\begin{center}
 \includegraphics{andrzej}
\end{center}
  \caption{Zdjęcie obecnego prezydenta Polski - Andrzeja Dudy}
\end{figure}
\bibliography{Biblio}
\end{document}